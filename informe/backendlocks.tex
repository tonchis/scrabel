\subsection{Locks}

Para asegurar la correcta sincronización entre clientes, creamos los siguientes \emph{rwlocks}:

\begin{itemize}

    \item uno para todo el tablero de palabras, al cual llamaremos $lockPalabras$;
    
    \item uno para cada casillero del tablero de palabras, por lo cual llamaremos $locksPalabras[i][j]$ al \emph{rwlock} correspondiente al casillero de la fila $i$ y columna $j$ del tablero de palabras;
    
    \item uno para cada casillero del tablero de letras, por lo cual llamaremos $locksLetras[i][j]$ al \emph{rwlock} correspondiente al casillero de la fila $i$ y columna $j$ del tablero de letras.

\end{itemize}

El motivo de esta desición la vamos a ir explicando en cada una de las peticiones de cliente que puede recibir el servidor. Por otro lado, utilizaremos las siguientes variables globales en los algoritmos:

\begin{itemize}

    \item una lista de casilleros llamada $palabra$ que representa las letras acumuladas que el usuario envió;
    
    \item una matriz de casilleros llamada $tableroLetras$ que representa las letras que todos los usuarios enviaron;
    
    \item una matriz de casilleros llamada $tableroPalabras$ que representa las letras de las palabras que todos los usuarios formaron.

\end{itemize}

\subsubsection{Letra}

Esta petición intenta poner una letra en el tablero de letras. El algoritmo original tiene el siguiente aspecto:

\begin{algorithm}[H]
\caption{letraRecibida($mensaje$)}
\begin{algorithmic}[1]
    \STATE $casillero \leftarrow$ decodificarCasilleroDeMensaje($mensaje$)
    \IF{esFichaValidaEnPalabra($casillero$, $palabra$)}
        \STATE $palabra \leftarrow palabra + letra(casillero)$.
        \STATE $tableroLetras[fila(casillero)][columna(casillero)] \leftarrow letra(casillero)$.
        \STATE enviar(OK).
    \ELSE
        \STATE quitarLetras($palabra$).
        \STATE enviar(ERROR).
    \ENDIF
\end{algorithmic}
\end{algorithm}

\noindent En esta parte simplemente optamos por usar el \emph{lock} del casillero en el tablero de letras al inicio del algoritmo en modo escritura, y lo liberamos al final. De este modo, cualquiera que desee leer o escribir el mismo casillero deberá esperar:

\begin{algorithm}[H]
\caption{letraRecibida'($mensaje$)}
\begin{algorithmic}[1]
    \STATE $casillero \leftarrow$ interpretarCasilleroDeMensaje($mensaje$)
    \STATE wlock($locksLetras[fila(casillero)][columna(casillero)]$).
    \STATE ...
    \STATE wunlock($locksLetras[fila(casillero)][columna(casillero)]$).
\end{algorithmic}
\end{algorithm}

Por otro lado, la función auxiliar original para indicar si una ficha es válida en una palabra:

\begin{algorithm}[H]
\caption{esFichaValidaEnPalabra($casillero$, $palabra$)}
\begin{algorithmic}[1]
    \IF{$casillero$ cae afuera del tablero}
        \RETURN INVALIDA.
    \ENDIF
    \IF{$tableroLetras[fila(casillero)][columna(casillero)] \neq \emptyset$}
        \RETURN INVALIDA.
    \ENDIF
    \IF{$palabra \neq \emptyset$}
        \IF{$casillero$ está alineado horizontalmente o verticalmente con $palabra$}
            \FOR{cada $casillero'$ entre $palabra$ y $casillero$}
                \IF{$casillero' \notin palabra$ y $tableroPalabras[fila(casillero')][columna(casillero')] = \emptyset$}
                    \RETURN INVALIDA.
                \ENDIF
            \ENDFOR
        \ELSE
            \RETURN INVALIDA.
        \ENDIF
    \ENDIF
    \RETURN VALIDA.
\end{algorithmic}
\end{algorithm}

\noindent En este caso estamos en un problema cuando queremos leer el tablero de palabras, asi que utilizamos el \emph{lock} de cada posición que queremos leer en modo lectura:

\begin{algorithm}[H]
\caption{esFichaValidaEnPalabra'($casillero$, $palabra$)}
\begin{algorithmic}[1]
    \STATE ...
    \FOR{cada $casillero'$ entre $palabra$ y $casillero$}
        \STATE rlock($locksPalabras[fila(casillero')][columna(casillero')]$).
        \IF{$casillero' \notin palabra$ y $tableroPalabras[fila(casillero')][columna(casillero')] = \emptyset$}
            \STATE runlock($locksPalabras[fila(casillero')][columna(casillero')]$).
            \RETURN INVALIDA.
        \ENDIF
        \STATE runlock($locksPalabras[fila(casillero')][columna(casillero')]$).
    \ENDFOR
    \STATE ...
\end{algorithmic}
\end{algorithm}

Por último, la función auxiliar para quitar todas las letras de una palabra:

\begin{algorithm}[H]
\caption{quitarLetras($palabra$)}
\begin{algorithmic}[1]
    \FOR{cada $casillero$ en $palabra$}
        \STATE $tableroLetras[fila(casillero)][columna(casillero)] \leftarrow \emptyset$.
    \ENDFOR
\end{algorithmic}
\end{algorithm}

\noindent Aquí simplemente utilizamos los \emph{locks} de cada casillero que queremos borrar en modo escritura:

\begin{algorithm}[H]
\caption{quitarLetras'($palabra$)}
\begin{algorithmic}[1]
    \FOR{cada $casillero$ en $palabra$}
        \STATE wlock($locksLetras[fila(casillero)][columna(casillero)]$).
        \STATE $tableroLetras[fila(casillero)][columna(casillero)] \leftarrow \emptyset$.
        \STATE wunlock($locksLetras[fila(casillero)][columna(casillero)]$).
    \ENDFOR
\end{algorithmic}
\end{algorithm}

\subsubsection{Palabra}

Esta petición pone todas las letras acumulada de la palabra actual del usuario en el tablero de letras. El algoritmo original tiene el siguiente aspecto:

\begin{algorithm}[H]
\caption{palabraRecibida()}
\begin{algorithmic}[1]
    \FOR{cada $casillero$ en $palabra$}
        \STATE $tableroPalabras[fila(casillero)][columna(casillero)] \leftarrow letra(casillero)$.
    \ENDFOR
    \STATE $palabra \leftarrow \emptyset$.
    \STATE enviar(OK).
\end{algorithmic}
\end{algorithm}

\noindent Para lograr que el pasaje de todas las letras no sea interrumpido por ningún motivo, utilizamos el lock completo del tablero de palabras para todo el algoritmo en mod escritura:

\begin{algorithm}[H]
\caption{palabraRecibida'()}
\begin{algorithmic}[1]
    \STATE wlock($lockPalabras$).
    \STATE ...
    \STATE wunlock($lockPalabras$).
\end{algorithmic}
\end{algorithm}

\subsubsection{Update}

Esta petición envia el estado del tablero de palabras. El algoritmo original tiene el siguiente aspecto:

\begin{algorithm}[H]
\caption{actualizacionRequerida()}
\begin{algorithmic}[1]
    \STATE $buffer \leftarrow \emptyset$
    \FOR{cada $casillero$ en $tableroPalabras$}
        \STATE $buffer \leftarrow buffer +$ codificarCasilleroParaMensaje($casillero$).
    \ENDFOR
    \STATE enviar($buffer$).
\end{algorithmic}
\end{algorithm}

\noindent De la misma manera que en el caso anterior, utilizamos el lock completo del tablero de palabras para todo el algoritmo, pero esta vez en modo lectura, para que cualquiera pueda pedir una actualización simultáneamente:

\begin{algorithm}[H]
\caption{actualizacionRequerida'()}
\begin{algorithmic}[1]
    \STATE rlock($lockPalabras$).
    \STATE ...
    \STATE runlock($lockPalabras$).
\end{algorithmic}
\end{algorithm}