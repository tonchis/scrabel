\section{RWLock}

Para resolver esta parte del trabajo se nos solicitó implementar un Read-Write Lock libre de inanición utilizando
únicamente semáforos POSIX, respetando la interfaz provista. Esto lo resolvimos implementando las funciones declaradas en \verb|RWLock.h|.

	Para garantizar que el sistema sea libre de inanición decidimos implementar una estructura que contiene una cola de lotes de lectores asociados a cada pedido de escritura. 
	La estructura funciona de la siguiente manera: cada vez que llega un pedido de \textit{lock} de escritura, se encola un lote con la información de éste y todos los pedidos de lectura que lleguen a continuación de él. En caso que luego de un pedido de \textit{lock} de escritura llegue otro de igual tipo, el lote asociado al primer  elemento encolado, será vacío.   


\begin{verbatim}
struct Batch {
  sem_t write_permission;
  sem_t reader
  s_sem;
  int readers;
  bool writed;
}
\end{verbatim}
	
\begin{algorithm}[H]
\caption{rlock()}
\begin{algorithmic}[1]
\STATE wait(mutex)
\STATE incrementamos la cantidad de lectores en una unidad
\IF{la cola de lotes está vacía}
\STATE signal(mutex)
\ELSE
\STATE agregamos el pedido de \textit{lock} de lectura al último lote de la cola
\STATE signal(mutex)
\STATE esperamos a que alguien nos avise que podemos leer 
\STATE avisamos a otro lector que puede leer
\ENDIF
\end{algorithmic}
\end{algorithm}


\begin{algorithm}[H]
\caption{runlock()}
\begin{algorithmic}[1]
\STATE wait(mutex)
\STATE decrementamos la cantidad de lectores en una unidad
\IF{la cola de lotes no está vacía}
	\STATE tomamos el \textit{lock} de escritura al que pertenece
	\IF{todavía hay pedidos de \textit{lock} de lectura en el lote}
		\STATE decrementamos la cantidad de los mismos en una unidad
	\ENDIF
	\IF{no hay pedidos de \textit{lock} de lectura en el lote}
		\IF{el pedido de escritura asociado al lote actual ya terminó}
			\STATE sacamos el lote de la cola
		\ENDIF
		\IF{la cola de lotes no está vacía}
			\STATE le avisamos al siguiente lote que ya puede escribir
		\ENDIF
	\ENDIF	 
\ENDIF
\STATE signal(mutex)

\end{algorithmic}
\end{algorithm}

\begin{algorithm}[H]
\caption{wlock()}
\begin{algorithmic}[1]
\STATE creamos un lote nuevo
\STATE wait(mutex)
\STATE encolamos el lote nuevo
\IF{la cola contiene un solo lote y no hay lectores}
\STATE dejamos que el lote nuevo escriba
\ENDIF

\STATE signal(mutex)
\STATE esperamos que el nuevo lote pueda escribir

\end{algorithmic}
\end{algorithm}

\begin{algorithm}[H]
\caption{wunlock()}
\begin{algorithmic}[1]
\STATE signal(mutex)
\STATE tomamos el primer lote de la cola
\STATE lo marcamos como que ya escribió
\IF{tiene lectores asociados}
\STATE los dejamos leer
%\STATE signal(mutex)
\ELSE
\STATE quitamos el lote de la cola
\IF{la cola no está vacía}
\STATE dejamos que el siguiente lote escriba
\ENDIF
%\STATE signal(mutex)
\ENDIF
\STATE signal(mutex)
\end{algorithmic}
\end{algorithm}


