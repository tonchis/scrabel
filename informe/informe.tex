\documentclass[a4paper,spanish]{article}
\usepackage[spanish]{babel}
\usepackage[utf8]{inputenc}
\usepackage{caratula}
\usepackage{amsmath, amscd, amssymb, amsthm, latexsym, verbatim}
\usepackage{graphicx, graphics, caption}
\usepackage{fancyhdr}
\usepackage{float, algorithmic}
\usepackage{hyperref}
\usepackage{algorithm}

%probando para los margenes
\usepackage[top=3cm, bottom=3cm, left=1cm, right=1cm]{geometry}

% configuro el paquete de algoritmos
\floatname{algorithm}{Algoritmo}

\makeatletter
\newcounter{algorithmic}
\let\ORIG@algorithmic\algorithmic
\def\algorithmic{\stepcounter{algorithmic}\ORIG@algorithmic}
\def\theHALC@line{\thealgorithmic-\theALC@line}
\def\theHALC@rem{\thealgorithmic-\theALC@rem}
\makeatother

% encabezados
\newcommand{\norma}[1]{\left|\left|#1\right|\right|}
\parskip=1ex
\pagestyle{fancy}
\pagenumbering{arabic}
%\fancyhf{}
\renewcommand{\headrulewidth}{0.02 cm}
\renewcommand{\footrulewidth}{0 cm}
\fancyhead[L]{Trabajo Pr\'actico N$^{\circ}$ 1}
\fancyhead[R]{Sistemas operativos}
\def\septad{\rule{16 cm}{.2 mm}}

% defino un environment propio para las ecuaciones
%\newenvironment{ecuacion}
%	{\begin{equation} \begin{aligned}}
%	{\end{aligned} \end{equation}}
%	
%\newenvironment{ecuacion*}
%	{\begin{equation*} \begin{aligned}}
%	{\end{aligned} \end{equation*}}

% comienzo el documento
\begin{document}

	\materia{Sistemas operativos}

\titulo{Trabajo Práctico 2}

\integrante{Ramiro Camino}{264/06}{ramirocamino@gmail.com}

\integrante{Gloria Josefina Diodati}{285/05}{josefa84@gmail.com}

\integrante{Lucas Pablo Tolchinsky}{591/07}{lucas.tolchinsky@gmail.com}

\maketitle

	
	\tableofcontents

	\section{Introducción}

Para este trabajo la cátedra nos presenta un webserver que corre el clásico juego del scrabble en un browser. El objetivo final del mismo es lograr que el juego sea multijugador, de manera de que muchos jugadores (browsers) compartan el mismo tablero, y puedan jugar y divertirse en familia.

Con esta idea, el trabajo se dividirá en dos partes: por un lado, controlar el acceso al tablero mediante locks implementados con semáforos POSIX y convertir el proceso del backend en un proceso multithread que utilize estos locks.

Sin más, y con ustedes, el TP.
%	\include{roundrobin}
%	\include{lottery}
	\section{Conclusión}
	%\begin{thebibliography}{9}

\bibitem {} Thomas H. Cormen, Charles E. Leiserson, Ronald L. Rivest, y Clifford Stein. Introduction to Algorithms, Second Edition. MIT Press and McGraw-Hill, 2001.

\bibitem {} Xumari, G.L. Introduction to dynamic programming. Wilwy $\&$ Sons Inc., New York. 1967.

\end{thebibliography}

	%palmitos, champignones a la provenzal 2, panceta y ciruela,
	%provolone 3, queso al oreganato 2, pollo 1
	
	%panceta? anana?

\end{document}