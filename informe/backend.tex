\section{Backend}

\subsection{Multithreading}

Para lograr que el backend del servidor sea mutlithreading, utilizamos la librería pthreads del sistema. Cada vez que el servidor acepta una nueva conexión, se levanta un nuevo thread apuntando a la función \textit{atendedor\_de\_jugador}. Para poder mantener control de los threads, tenemos como variable global un \textit{map} de sockets a threads.

\ 
\begin{algorithmic}
\IF{(socketfd\_cliente = accept(socket\_servidor, ...)) == -1}
  \STATE error: ``Error al aceptar la conexión.''
\ELSE
  \STATE pthread\_t tid
  \STATE pthread\_create(\&tid, NULL, atendedor\_de\_jugador, (void*) socketfd\_cliente)
  \STATE threads[socketfd\_cliente] = \&tid
\ENDIF
\end{algorithmic}
\

Luego, cuando una conexión se interrumpe, debemos cerrar el thread:
\
\begin{algorithm}[H]
\caption{$terminar\_servidor\_de\_jugador(socket\_fd, palabra\_actual)$}
\begin{algorithmic}[1]
\STATE close(socket\_fd)
\STATE quitar\_letras(palabra\_actual)
\STATE pthread\_exit((void*) -1)
\STATE threads.erase(socket\_fd)
\end{algorithmic}
\end{algorithm}

Además, cuando se cierra el servidor, hay que cerrar todos los threads que quedan sueltos.
\
\begin{algorithm}[H]
\caption{$cerrar\_servidor(signal)$}
\begin{algorithmic}
\STATE close(socket\_servidor)
\FOR{thread \textbf{en} threads}
  \STATE pthread\_kill(thread, 9)
\ENDFOR
\STATE threads.clear()
\end{algorithmic}
\end{algorithm}
